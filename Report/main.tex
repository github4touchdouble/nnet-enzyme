\documentclass{bioinfo}
\copyrightyear{2015} \pubyear{2015}

\access{Advance Access Publication Date: Day Month Year}
\appnotes{Manuscript Category}

\begin{document}
\firstpage{1}

\subtitle{Subject Section}

\title[short Title]{This is a title}
\author[Sample \textit{et~al}.]{Corresponding Author\,$^{\text{\sfb 1,}*}$, Co-Author\,$^{\text{\sfb 2}}$ and Co-Author\,$^{\text{\sfb 2,}*}$}
\address{$^{\text{\sf 1}}$Department, Institution, City, Post Code, Country and \\
$^{\text{\sf 2}}$Department, Institution, City, Post Code,
Country.}

\corresp{$^\ast$To whom correspondence should be addressed.}

\history{Received on XXXXX; revised on XXXXX; accepted on XXXXX}

\editor{Associate Editor: XXXXXXX}

\abstract{\textbf{Motivation:} \\
	\textbf{Results:}\\
\textbf{Availability:} \\
\textbf{Contact:} \href{name@bio.com}{name@bio.com}\\
\textbf{Supplementary information:} Supplementary data are available at \textit{Bioinformatics}
online.}

\maketitle

\section{Abstract}

The accurate prediction of enzyme commission numbers (EC numbers) is not only crucial for 
the classification and understanding of newly discovered enzymes but also for completing the annotation of already known enzymes.
Therefore, developing a reliable method for predicting EC numbers is of great importance.

However, due to insufficient data, enzyme function prediction using machine learning is an ongoing challenge.
In this paper, we propose several methods for predicting enzymes in three different problem categories (Table~\ref{Tab:01}).
Throughout the developing of our models, we used a variety of different input features and machine learning algorithms, of which the best will be thoroughly reviewed in this paper.
\begin{center}
\begin{table}[!htbp]
\processtable{Description of subproblem categories\label{Tab:01}} {\begin{tabular}{@{}llll@{}}\toprule 
		Level & Description & Best performing method & F1 score\\\midrule
		0 & Binary classification & Random Forest & score\\
		1 & Main class classification & Feedforward neuronal network & score \\
		2 & Subclass classification & Feedforward neuronal network & score \\\botrule
\end{tabular}}{}
\end{table}
\end{center}




\section{Introduction}

Text Text Text Text Text Text  Text Text.  Text Text Text
%\enlargethispage{12pt}

\section{Approach}
Text Text Text Text Text Text  Text Text.  Text Text Text

\begin{methods}
\section{Methods}
Text Text Text Text Text Text  Text Text.  Text Text Text


\begin{itemize}
\item for bulleted list, use itemize
\item for bulleted list, use itemize
\item for bulleted list, use itemize\vspace*{1pt}
\end{itemize}

\subsection{This is subheading}
Text Text Text Text Text Text  Text Text.  Text Text Text

\subsubsection{This is subsubheading}
Text Text Text Text Text Text  Text Text.  Text Text Text


\end{methods}

%\centerline{\includegraphics{fig01.eps}}

%\begin{figure}[!tpb]%figure2
%%\centerline{\includegraphics{fig02.eps}}
%\caption{Caption, caption.}\label{fig:02}
%\end{figure}
\section{Discussion}
Text Text Text Text Text Text  Text Text.  Text Text Text 

\section{Conclusion}
Text Text Text Text Text Text  Text Text.  Text Text Text

\begin{enumerate}
\item this is item, use enumerate
\item this is item, use enumerate
\item this is item, use enumerate
\end{enumerate}

\section*{Acknowledgements}


\section*{Funding}

This work has been supported by the... Text Text  Text Text.\vspace*{-12pt}

\end{document}
